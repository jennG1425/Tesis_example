\documentclass[12pt]{report}
\usepackage[a6paper,margin=2in]{geometry}
  \newgeometry{
    textheight=9in,
    textwidth=6.5in,
    top=1in,
    headheight=14pt,
    headsep=25pt,
    footskip=30pt
  }
\usepackage[utf8]{inputenc}
\usepackage{graphicx}
\usepackage{fancyhdr}
\fancyhf{}
\pagestyle{fancy}
\renewcommand{\headrulewidth}{0.4pt}
\fancyheadoffset{0pt}
\rhead{\scshape \footnotesize }
\chead{}
\cfoot{\thepage}
\usepackage{natbib}
\usepackage{gensymb}
\usepackage{longtable}
\usepackage{adjustbox}
\usepackage{lipsum}
\usepackage{multirow}
\usepackage[table,xcdraw]{xcolor}
\definecolor{Amarillo}{HTML}{F39C12}
\definecolor{Azul}{HTML}{2874A6}
\definecolor{Rojo}{HTML}{E74C3C}
\usepackage{wrapfig}
\usepackage{pdfpages}
\usepackage{threeparttable}
\usepackage{subcaption}
\usepackage{enumerate}
\usepackage{booktabs}
\usepackage{dirtytalk}
\usepackage[nottoc]{tocbibind}
\usepackage[spanish, mexico]{babel}
\usepackage{hyperref}
\usepackage{amsmath}
\usepackage{physics}
\usepackage{amsfonts} 
\usepackage{float}
\usepackage{color}
\usepackage{tcolorbox}
\usepackage{listings} %Para agregar un código en el lenguaje de tu preferencia
\lstset{ %
language=Mathematica,                % choose the language of the code
basicstyle=\footnotesize,       % the size of the fonts that are used for the code
numbers=left,                   % where to put the line-numbers
numberstyle=\footnotesize,      % the size of the fonts that are used for the line-numbers
stepnumber=1,                   % the step between two line-numbers. If it is 1 each line will be numbered
numbersep=5pt,                  % how far the line-numbers are from the code
backgroundcolor=\color{white},  % choose the background color. You must add \usepackage{color}
showspaces=false,               % show spaces adding particular underscores
showstringspaces=false,         % underline spaces within strings
showtabs=false,                 % show tabs within strings adding particular underscores
frame=single,           % adds a frame around the code
tabsize=2,          % sets default tabsize to 2 spaces
captionpos=b,           % sets the caption-position to bottom
breaklines=true,        % sets automatic line breaking
breakatwhitespace=false,    % sets if automatic breaks should only happen at whitespace
escapeinside={\%*}{*)}          % if you want to add a comment within your code
}


\begin{document}

% ----- I made the cover in pages/word and exported it to pdf to look like how i wanted 
\includepdf{Portada LIER}

% ------------------  TABLE OF CONTENTS --------------------
\tableofcontents 

% -------------------  LIST OF FIGURES --------------------

%{\let\oldnumberline\numberline       % Uncomment to add the word 'Figure' to figure number in List of Figures
%\renewcommand{\numberline}{\figurename~\oldnumberline}  
\listoffigures%}


% -------------------  LIST OF TABLES ---------------------

\listoftables 


\chapter*{}
\begin{flushright}
\textit{Dedicatoria, ¿a quien le dedicas tu tesis?}
\end{flushright}


% incio de la tesis

\chapter*{Agradecimientos}
\addcontentsline{toc}{chapter}{Agradecimientos}
\input{chapters/Agredecimientos}

\chapter*{Resumen}
\addcontentsline{toc}{chapter}{Resumen}
\input{chapters/resumen}

\chapter{Antecedentes}
\input{chapters/Antecedentes}


\chapter{Metodología}
\input{chapters/Metodología}

\chapter{Resultados}
\input{chapters/Resultados}

\chapter{Conclusiones}
\input{chapters/Conclusiones}

%Si quieres añadir un codigo debes de usar \usepackage{listings}
\appendix
\chapter{Código en Mathematica}
\begin{lstlisting}
Newton

In[370]:= NwAJ1 = NonlinearModelFit[wAJ1, Exp[-k  r], k, r, Method -> NMinimize];
NwAJ1["BestFitParameters"]
NwAJ1["ParameterTable"];
vNwAJ1 = NwAJ1[{"MeanPredictionConfidenceIntervalTable", 
    "SinglePredictionConfidenceIntervalTable"}];
Normal[vNwAJ1[[1, 1]]];

Out[371]= {k -> -0.0134066}

Henderson and Pabis

In[229]:= HPwAJ1 = NonlinearModelFit[wAJ1, a Exp[-(k r)], {a, k}, r, 
   Method -> NMinimize];
HPwAJ1["BestFitParameters"]
HPwAJ1["ParameterTable"];
vHPwAJ1 = HPwAJ1[{"MeanPredictionConfidenceIntervalTable", 
    "SinglePredictionConfidenceIntervalTable"}];
Normal[vHPwAJ1[[1, 1]]];

Out[230]= {a -> 218.825, k -> 0.00135116}

Logarithmic

In[235]:= LwAJ1 = NonlinearModelFit[wAJ1, a Exp[-(k x)] + s, {a, k, s}, x, 
   Method -> NMinimize];
LwAJ1["BestFitParameters"]
LwAJ1["ParameterTable"];
vLwAJ1 = LwAJ1[{"MeanPredictionConfidenceIntervalTable", 
    "SinglePredictionConfidenceIntervalTable"}];
Normal[vLwAJ1[[1, 1]]];

Out[236]= {a -> 162.717, k -> 0.00223173, s -> 60.7403}

Two-term

In[328]:= TtwAJ1 = NonlinearModelFit[wAJ1, a Exp[-(k r)] + v Exp[-k1 r], {a, k, v, k1}, 
   r, Method -> NMinimize];
TtwAJ1["BestFitParameters"]
TtwAJ1["ParameterTable"];
vTtwAJ1 = TtwAJ1[{"MeanPredictionConfidenceIntervalTable", 
    "SinglePredictionConfidenceIntervalTable"}];
Normal[vTtwAJ1[[1, 1]]];

Out[329]= {a -> 104.388, k -> 17.2617, v -> 214.118, k1 -> 0.00126168}


Two-term exp

In[334]:= TtEwAJ1 = NonlinearModelFit[wAJ1, 
   a Exp[-(k r)] + (1 - a) Exp[-k1 r], {a, k, k1}, r, Method -> NMinimize];
TtEwAJ1["BestFitParameters"]
TtEwAJ1["ParameterTable"];
vTtEwAJ1 = 
  TtEwAJ1[{"MeanPredictionConfidenceIntervalTable", 
    "SinglePredictionConfidenceIntervalTable"}];
Normal[vTtEwAJ1[[1, 1]]];
Out[335]= {a -> -210.097, k -> 6.40243, k1 -> 0.00120377}

During evaluation of In[334]:= General::stop: Further output of General::munfl will be suppressed during this calculation.

Midilli and Kukuck

In[253]:= MKwAJ1 = NonlinearModelFit[wAJ1, a Exp[-(k r)] + v r, {a, k, v}, r, 
   Method -> NMinimize];
MKwAJ1["BestFitParameters"]
MKwAJ1["ParameterTable"];
vMKwAJ1 = MKwAJ1[{"MeanPredictionConfidenceIntervalTable", 
    "SinglePredictionConfidenceIntervalTable"}];
Normal[vMKwAJ1[[1, 1]]];

Out[254]= {a -> 278.421, k -> 0.00692511, v -> 0.363637}

Wang and Singh

In[259]:= WwAJ1 = NonlinearModelFit[wAJ1, 1 + a r + b r^2, {a, b}, r, 
   Method -> NMinimize];
WwAJ1["BestFitParameters"]
WwAJ1["ParameterTable"];
vWwAJ1 = WwAJ1[{"MeanPredictionConfidenceIntervalTable", 
    "SinglePredictionConfidenceIntervalTable"}];
Normal[vWwAJ1[[1, 1]]];

During evaluation of In[259]:= NonlinearModelFit::lmnl: The model 1+a r+b r^2 is linear in the parameters {a,b}, but a nonlinear method or non-Euclidean norm was specified, so nonlinear methods will be used.

Out[260]= {a -> 1.58327, b -> -0.00330745}

Primer Orden

In[265]:= POwAJ1 = NonlinearModelFit[wAJ1, (a + (b* r))/(1 + (c * r)), {a, b, c}, r, 
   Method -> NMinimize];
POwAJ1["BestFitParameters"]
POwAJ1["ParameterTable"];
vPOwAJ1 = POwAJ1[{"MeanPredictionConfidenceIntervalTable", 
    "SinglePredictionConfidenceIntervalTable"}];
Normal[vPOwAJ1[[1, 1]]];

Out[266]= {a -> 346.835, b -> 6.63127, c -> 0.0491346}

Segundo Orden

In[271]:= SOwAJ1 = NonlinearModelFit[wAJ1, (a + (b* r) + (c*( r^2)))/(
   1 + (d r) + ( e (r^2))), {a, b, c, d, e}, r, Method -> NMinimize];
SOwAJ1["BestFitParameters"]
SOwAJ1["ParameterTable"];
vSOwAJ1 = SOwAJ1[{"MeanPredictionConfidenceIntervalTable", 
    "SinglePredictionConfidenceIntervalTable"}];
Normal[vSOwAJ1[[1, 1]]];

Out[272]= {a -> 328.87, b -> 25.6664, c -> 0.756661, d -> 0.0728934, e -> 0.00540491}

\end{lstlisting}



% Solo cambiar las referencias, la forma en la que se añaden es en formato APA
\bibliographystyle{references/models-2names}
\bibliography{references/references}

\end{document}
